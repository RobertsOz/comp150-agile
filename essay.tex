
% Please do not change the document class
\documentclass{scrartcl}

% Please do not change these packages
\usepackage[hidelinks]{hyperref}
\usepackage[none]{hyphenat}
\usepackage{setspace}
\doublespace

% You may add additional packages here
\usepackage{amsmath}

% Please include a clear, concise, and descriptive title
\title{How does the product owner relationship with the team differ between software and games development?}

% Please do not change the subtitle
\subtitle{COMP150 - Agile Development Practice}

% Please put your student number in the author field
\author{1606695}

\begin{document}

\maketitle

\abstract{In this paper I compare software development and games development to find out what the differences between product owner roles are in an agile work-flow, to do that I use academic papers and thoughts from industry influencers.}

\section{Introduction}

With this essay I would like to familiarise people to the key differences between product owners in software and games development. Knowing what kind of a product owner your team has could help you avoid situations where you get into a contract that will have you doing extra work just because your product owner is bad at scoping the project. What kind of pro¬duct owner situations work for software development but not in the games industry?  
As Ken H. Judy and Ilio Krumins-Beens state in their paper: “Great scrums need great product owners” [1]. And because product owners have an important role in deciding how the project will go, it is extremely important to understand what are your options when you are in this kind of situation, so you can correct the course of the project.

\section{The upsides and down sides of Software development.}

One of the biggest pitfalls in the software industry is when a product owner does not clearly define what the product should do. Ben Horowitz and David Weiden define good product owners as having “clear, written communication with product development” [2]. This is why product owners are commonly lead users of the system, someone from marketing or anyone with a understanding of users, the market place, the competition and of future trends [3]. If you have no understanding of the software development process you cannot understand what the scope for a certain feature would be and when that feature should be introduced. What ends up happening with bad product owners is that they remember a feature that they would like to be added to the project in the middle or even sometimes at the end of the project, and now the deadline is approaching rapidly and your team is in a scramble working overtime trying to add this feature. And all of this extra stress and overtime work could have been avoided if the product owner would have laid out a clear plan for the project from the beginning. In a situation like this what you should do is demand the project specifics to be written out at the very beginning so your team could adjust the scope of the project accordingly.  
And this in turn is one of the biggest advantages and disadvantages in the software industry you are often hired do these very specific software jobs and the finished product has to have all the features laid out at the beginning of the project.

\section{The upsides and down sides of Games development.}

Games development is a creative industry and a good product owner lets the creative people be creative as Clinton Keith writes in a Gamasutra article - “Microsoft, our publisher, and the studio we worked at, Angel Studios, left us largely alone to develop the game” [4]. But with the games industry growing rapidly publishers cannot afford to take risks with no oversight over the game. That is why a common solution is to divide up the product owner roles into a publisher product owner and a developer product owner [5]. In the games industry a product owner represents the player and he has to foresee what the market will embrace up to three years in advance [6]. That is what the publisher product owner does, he will analyse the market and keep the game “on track” to reach a certain goal. While the developer product owner is making the game with the team and is there every day to help refine the sprint goal and remains the voice for the team.
But the main thing about games development is you could set out to make one game and end up with a completely different one. The game development process is an evolving one the game could change drastically between scrums and a good product owner will help these drastic changes along and transition them into finished products. 

\section{What kind of personality suits software and games development?}

The structure for software development and games development on the surface is very similar, you work on strict deadlines, you can be hired to work on something and you can choose to make something on your own/with your own team. The major difference is the deliver-ables have completely different parameters, when you deliver software it either works and has every feature promised or it does not. On the other hand in games development the deliverable could be subjective the goals set out before the project started could have changed while still keeping the core of the product like story, characters. So if you are a person who needs this structure involving clear and defined goals, software development is a great choice for you. Though if you thrive in ambiguity you will be very successful in games development, throughout the project you could be working on something completely new that was not set out at the beginning of the project.
\section{Conclusion}

Comparing the most common scenarios of product owners between software and games development, I conclude that a software product owner is usually a client/a company while in the games industry the product owner is often the user/person who will be playing the game. This is because software has a purpose it has to do something for someone, while a game is subjective on what its purpose is and if that purpose was accomplished, either telling a story or teaching something.




\bibliographystyle{ieeetran}
\bibliography{references}
\section{Referances}
\section*{Paper 1}

\begin{description}
\item[Title:] Great Scrums Need Great Product Owners: Unbounded Collaboration  and Collective Product Ownership
\item[Citation:] “Great scrums need great product owners” [1]
\item[Abstract:] ``Scrum describes a separation of roles; the product owner is accountable for achieving business objectives and the team for technical execution. A pragmatic and collegial relationship between a product owner and team can satisfy the definition of collaboration and honor roles while barely tapping or actually working against the potential of a project and its participants. This paper surveys literature to describe different forms of collaboration, to establish that deep, unbounded collaboration is at the heart of agile values, and that partnerships of high trust and shared risk lead to value and innovation. Finally, this paper incorporates a real- world example of a product owner who, while remaining accountable to the outcome, shared ownership over vision, priorities and execution with her scrum/XP development team.''
\item[Web link:] \url{http://ieeexplore.ieee.org.ezproxy.falmouth.ac.uk/xpls/icp.jsp?arnumber=4439168&tag=1}
\item[Full text link:] \url{http://ieeexplore.ieee.org.ezproxy.falmouth.ac.uk/xpls/icp.jsp?arnumber=4439168&tag=1#article}
\item[Comments:] I was looking for the importance of a good product owner.
\end{description}

\section*{Paper 2}
\begin{description}
\item[Title:] Title of paper
\item[Citation:] “clear, written communication with product development” [2]
\item[Abstract:] A Good Product Manager plays critical role in a successful product.
A successful product is 
the highest impact contribution that anyone can
make in the PD organization.
In fact, the 
number one criteria for selecting a Vice President is the candidate's track record (or lack 
thereof) of successful products that become profitable businesses for their company. 
Being a good product manager is so
hard that most product managers at most companies fail to 
be good 
--
and instead are bad.
Because product management is a highly leveraged position, a 
bad product manager leads to many other bad consequences, generally including the wrong 
product being built, which generally has a significant impact on revenue, morale, and 
reputation 
--
of both the product manager and their company. 
There are a number of straightforward principles that product managers can follow which will 
dramatically increase their chance of success.
Surprisingly, only very few product managers 
follow these principles.
Part of the problem is that these principles often are not articulated 
clearly, which this document attempts to address. 
A final note is that product management is a 
demanding and high profile job.
Individuals 
should make sure they're up to the challenge. 
\item[Web link:] \url{http://universalrg.org/FullText/2013153.pdf}
\item[Full text link:] \url{http://universalrg.org/FullText/2013153.pdf}
\item[Comments:] I was looking for what makes a good product owner in the software development industry and found this paper.
\end{description}

\section*{Paper 3}
\begin{description}
\item[Title:] Product Owner
\item[Citation:] This is why product owners are commonly lead users of the system, someone from marketing or anyone with a understanding of users, the market place, the competition and of future trends [3]
\item[Abstract:] The Scrum product owner is typically a project's key stakeholder. Part of the product owner responsibilities is to have a vision of what he or she wishes to build, and convey that vision to the scrum team. This is key to successfully starting any agile software development project. The agile product owner does this in part through the product backlog, which is a prioritized features list for the product.
\item[Web link:]\url{https://www.mountaingoatsoftware.com/agile/scrum/product-owner}
\item[Comments:] I was looking for a software company talking about product owners and found Mountain Goat Software
\end{description}


\section*{Paper 4}
\begin{description}
\item[Title:] Agile Game Development With Scrum: Teams
\item[Citation:] “Microsoft, our publisher, and the studio we worked at, Angel Studios, left us largely alone to develop the game” [4]
\item[Abstract:] I've worked on creating various products, from the F-22 fighters to games, for more than 20 years. The highlights of my career are clearly marked in my mind by the project teams I was working with. These teams were more consequential to enjoyment and productivity than the company or project we were working for at the time.
\item[Web link:] \url{http://www.gamasutra.com/view/feature/6040/agile_game_development_with_scrum_.php?print=1}
\item[Comments:] I was looking for examples of product owners in the games industry and found this extract from Clinton Keiths book "Agile Game Development with Scrum" on Gamasutra
\end{description}

\section*{Paper 5}
\begin{description}
\item[Title:] Publisher side product owners
\item[Citation:] That is why a common solution is to divide up the product owner roles into a publisher product owner and a developer product owner [5]
\item[Abstract:] Product owners for a video game project using Scrum are usually a member of the development team. The reason for this is that unlike projects outside the game industry, product owners are needed to provide a much higher level of subjective feedback. Does the control of the player feel right? Is a mechanic “fun enough”? This feedback requires daily engagement with the team.
\item[Web link:] \url{http://blog.agilegamedevelopment.com/2009/04/publisher-side-product-owners.html}
\item[Comments:] I was looking for examples of the product owner dynamic withing a games company and found this blog post by Clinton Keith.
\end{description}

\section*{Paper 6}
\begin{description}
\item[Title:] Agile Game Development with Scrum
\item[Citation:] In the games industry a product owner represents the player and he has to foresee what the market will embrace up to three years in advance [6]
\item[Abstract:] Product owners for a video game project using Scrum are usually a member of the development team. The reason for this is that unlike projects outside the game industry, product owners are needed to provide a much higher level of subjective feedback. Does the control of the player feel right? Is a mechanic “fun enough”? This feedback requires daily engagement with the team.
\item[Web link:] \url{https://github.com/zhang-yan-talendbj/book/blob/master/Agile%20Game%20Development%20with%20Scrum.pdf}
\item[Comments:] I was looking at what the product owners main responsibilities are in the games industry and found this book.
\end{description}
\end{document}
